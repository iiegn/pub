\documentclass[11pt,a4paper]{moderncv}

% moderncv themes
\moderncvtheme[blue]{classic}                 % optional argument are 'blue' (default), 'orange', 'red', 'green', 'grey' and 'roman' (for roman fonts, instead of sans serif fonts)
%\moderncvtheme[green]{classic}                % idem

% character encoding
\usepackage[utf8]{inputenc}                   % replace by the encoding you are using

\usepackage{url}
%% Define a new 'leo' style for the package that will use a smaller font.
\makeatletter
\def\url@leostyle{%
  \@ifundefined{selectfont}{\def\UrlFont{\sf}}{\def\UrlFont{\footnotesize\color{see}\itshape}}}

\makeatother
%% Now actually use the newly defined style.
\urlstyle{leo}

% personal data (the given example is exhaustive; just give what you want)
\firstname{egon~w.}
\familyname{stemle}
%\title{Cognitive Scientist}
\address{Institute of Cognitive Science\\University of Osnabr\"{u}ck\\Albrechtstr. 28}{49076 Osnabr\"{u}ck} % for classic style
%\address{Sedanstrasse~16, 49076~Osnabr\"{u}ck, Germany}  % for casual style
\phone{+49.541.969-3362}
\email{egon.stemle@uos.de}
%\extrainfo{\url{http://www.iiegn.de}}
\photo[64pt]{estemle} % also optional, and the optional argument is the height the picture must be resized to

%\renewcommand{\listsymbol}{{\fontencoding{U}\fontfamily{ding}\selectfont\tiny\symbol{'102}}} % define another symbol to be used in front of the list items

% the ConTeXt symbol
\def\ConTeXt{%
  C%
  \kern-.0333emo%
  \kern-.0333emn%
  \kern-.0667em\TeX%
  \kern-.0333emt}

% slanted small caps (only with roman family; the sans serif font doesn't exists :-()
%\usepackage{slantsc}
%\DeclareFontFamily{T1}{myfont}{}
%\DeclareFontShape{T1}{myfont}{m}{scsl}{ <-> cork-lmssqbo8}{}
%\usefont{T1}{myfont}{m}{scsl}Testing the font

% command and color used in this document, independently from moderncv 
\definecolor{see}{rgb}{0.5,0.5,0.5}% for web links
\newcommand{\up}[1]{\ensuremath{^\textsf{\scriptsize#1}}}% for text subscripts

%----------------------------------------------------------------------------------
%            content
%----------------------------------------------------------------------------------
\begin{document}
\maketitle

\section{}
Egon Stemle is a Cognitive Scientist: he studies skills like perception, thinking, learning, motor function, and language by combining the humanistic and analytical methods of the arts and the formal sciences.

His research focus within this relatively new `inter-discipline' lies in the area where Computational Linguistics and Artificial Intelligence converge. He is working on computer aided fabrication of ontologies from large document repositories, the technological feasibility thereof and the utilization of cross-linked structured data in applications, and on tools for editing, processing, and annotating linguistic data.

His driving force is the question why humans handle incomplete and -- more often than not -- inconsistent structured concepts just fine, whereas computational processes are often of little avail or fail completely. 


\section{Education}


	\cventry{02.2006 --\\-- 03.2009}{MSc in Cognitive Science}{}{}{University of Osnabr\"{u}ck, Germany}{Majors: \emph{Linguistics and Computational Linguistics} and \emph{Artificial Intelligence}, Study Project: ASADO (Analysis and Structure of Aviation Documents), [\url{http://www.cogsci.uni-osnabrueck.de/~ASADO}]. }
	\cventry{03.2002 --\\-- 09.2002}{Semester abroad}{ILLC (Institiute for Logic, Language and Computation)}{}{University of Amsterdam, The Netherlands}{}
	\cventry{10.2000 --\\-- 01.2006}{BSc in Cognitive Science}{}{}{University of Osnabr\"{u}ck, Germany}{}
	\cventry{12.1998 --\\-- 07.2000}{Apprenticeship as Travel Agent}{}{}{WORLDWIDE Touristik, N\"{u}rnberg, Germany}{}
	\cventry{06.1997}{Abitur}{}{}{Sigmund-Schuckert-Gymnasium, N\"{u}rnberg, Germany}{}
\closesection{}

\section{Master's Thesis}
	\cvitem{title}{\emph{Hybrid Sweeping: Streamlined Perceptual Structured-Text Refinement}}
	\cvitem{supervisors}{Prof.~Dr.~Stefan Evert and Prof.~Dr.~Peter K\"{o}nig}
	\cvitem{description}{\small Development of a perceptually driven content extraction architecture for Web pages -- exemplified by a Web page cleaning system.}
\closesection{}

\section{Bachelor's Thesis}
	\cvitem{title}{\emph{A Flexible Integration of CL Techniques for Analyzing Linguistic Units in Large Document Repositories}}
	\cvitem{supervisors}{PD Petra Ludewig and PD Helmar Gust}
	\cvitem{description}{\small The basic computational linguistic processing stages from raw text documents
towards a linguistically enriched annotation for documents are presented.}
\closesection{}

\pagebreak

\section{Research and Professional Experience}
	%\cventry{date}{main title}{three}{four}{five}{six}
	%\cventry{05.2008}{Freelancer}{ontoprise GmbH [\url{http://www.ontoprise.de}]}{four}{OmniFind, OntoBroker, SemanticMiner}{six}
	%\cventry{10.2008}{Freelancer}{ontoprise GmbH}{four}{Machbarkeitsstudie}{six}
	\cventry{since 04.2009}{Research Fellow}{LiveMemories sponsored Grant, financed by the 
Provincia Autonoma of Trento, for research activities at the Center for Mind/Brain sciences (CIMeC) 
of the University of Trento [\url{http://www.livememories.org}],[\url{http://portale.unitn.it/cimec}]}{}{}{}
	\cventry{2008}{Freelancer}{ontoprise GmbH [\url{http://www.ontoprise.de}]}{}{}{}
	\cventry{since 04.2008}{Project Member}{GoodGaze [\url{http://www.goodgaze.com}]}{Research Project to predict where People will look on Web pages}{}{}
	\cventry{10.2006 --\\-- 04.2008}{Research Participant}{DAAD funded research project (Project-linked exchange of academics and scientists, PPP)}{Collaboration between the Institute of Cognitive Science, U.~Osnabr\"{u}ck, Germany, the CNRS UMR 8163 (Savoirs, Textes, Langages), U.~of Lille (III), France, and the Departament de Traducci\'{o} i Filologia, U.~Barcelona (Pompeu Fabra), Spain}{Reference to Abstract Objects in Natural Language (OntoRef) [\url{http://parles.upf.es/glif/pub/abstract/index.html}]}{}
	\cventry{04.2006 --\\-- 04.2008}{Research Assistant}{Institute of Applied Informatics and Formal Description Methods, University of Karlsruhe, Germany}{Knowledge Management Research Group}{[\url{http://www.aifb.uni-karlsruhe.de/Forschungsgruppen/WBS/english}]}{Maintenance and development of software and scientific counsel; e.g.~evaluating CL tools for OmniFind+UIMA and hooking them together. }
	\cventry{08.2004 --\\-- 10.2005}{Research Participant}{Cooperation between the Universities of Osnabr\"{u}ck and Hildesheim, and the Aircraft Manufacturer AIRBUS}{Project to research Methodologies and Technologies to analyze and structure the huge Amount of Documentation produced during Aircraft Construction}{}{}
	\cventry{08.2003 --\\-- 12.2003}{Student Assistant}{Institute of Cognitive Science, University of Osnabr\"{u}ck, Germany}{Artificial Intelligence Research Group}{[\url{http://www.cogsci.uni-osnabrueck.de/~ai/milca.html}]}{Maintenance and development of software for the MiLCA (Media intensive learning units for courses in computational linguistics) project.}
	\cventry{since 10.2001}{System Administrator}{Institute of Cognitive Science, University of Osnabr\"{u}ck, Germany}{}{}{Maintenance of computer hardware and software, user support, server administration, configuration of a fully automated installation system for Windows, Linux and OS~X.}
	\cventry{11.1997 --\\-- 11.1998}{Alternative Civilian Service}{Alten-, Wohn- und Pflegeheim Reichelsdorf, and Fahrdienst f\"{u}r Behinderte gGmbH; N\"{u}rnberg, Germany}{}{}{}
\closesection{}

%\pagebreak

\section{Teaching Experience}
	\cventry{10.2003 --\\-- 02.2006\\(every other semester)}{Artificial Intelligence Tutor}{Methods of Artificial Intelligence}{Second year Cognitive Science class}{Institute of Cognitive Science, University of Osnabr\"{u}ck}{Weekly tutoring of the lecture material, assessment of programming assignments, homework corrections, and exam preparation.}	
	\cventry{04.2003 --\\-- 07.2006\\(every other semester)}{Computational Linguistics Tutor}{Introduction to Computational Linguistics}{First year Cognitive Science class}{Institute of Cognitive Science, University of Osnabr\"{u}ck}{Weekly tutoring of the lecture material, preparation and assessment of programming assignments, homework corrections, and exam preparation.}
	\cventry{10.2002 --\\-- 02.2003}{Theoretical Neuroscience Tutor}{Introduction to Theoretical Neuroscience}{Second year Cognitive Science class}{Institute of Cognitive Science, University of Osnabr\"{u}ck}{Weekly tutoring of the lecture material, homework corrections.}
	\cventry{10.2001 --\\-- 02.2002}{Computer Science Tutor}{Algorithms}{First year Computer Science class}{Computer Science Department, University of Osnabr\"{u}ck}{Weekly oral assessment of 10 2-people groups and homework corrections.}
\closesection{}

% \pagebreak

\section{Workshops and Conferences}
	\cventry{09.2006}{Student Worker at OTT06}{Workshop}{jointly organized by the Institute of Cognitive Science at the University of Osnabr\"{u}ck and the project C2 of the distributed DFG-research group Text Technological Modelling of Information}{[\url{http://www.cogsci.uni-osnabrueck.de/~ott06}]}{Ontologies in Text Technology: Approaches to Extract Semantic Knowledge from Syntactic Information.}

	\cventry{06.2006}{Student Worker at QITL-2}{Workshop}{organized by the Computational Linguistics Group at the Institute of Cognitive Science}{[\url{http://www.cogsci.uni-osnabrueck.de/~qitl}]}{Second Workshop on Quantitative Investigations in Theoretical Linguistics.}

	\cventry{09.2003}{Student Worker at EuroCogSci 2003}{Conference}{jointly organized by the Cognitive Science Society and the German Cognitive Science Society}{[\url{http://www.eurocogsci03.uni-osnabrueck.de}]}{First European Cognitive Science Meeting.}
\closesection{}

%\pagebreak

%\nocite{*}
%\bibliographystyle{plain}
%\bibliography{publications}
% see CV_en.bbl and modify cventries accordingly...

\section{Publications}
\cventry{09.2007}{Daniel Bauer, Judith Degen, Xiaoye Deng, Priska Herger, Jan Gasthaus, Eugenie Giesbrecht, Lina Jansen, Christin Kalina, Thorben Kr{\"u}ger, Robert M{\"a}rtin, Martin Schmidt, Simon Scholler, Johannes Steger, Egon Stemle, and Stefan Evert}{FIASCO: Filtering the internet by automatic subtree classification, Osnabr{\"u}ck}{in {\em Building and Exploring Web Corpora. Proceedings of the 3rd Web as Corpus Workshop, incorporating CLEANEVAL (WAC3-2007)}}{Presses universitaires de Louvain}{[\url{http://purl.org/stefan.evert/PUB/BauerEtc2007_FIASCO.pdf}]}

\cventry{07.2007}{Sebastian Blohm, Philipp Cimiano, and Egon Stemle}{Harvesting relations from the web - quantifiying the impact of filtering functions}{in {\em Proceedings of the 22nd Conference on Artificial Intelligence (AAAI-07)}}{Association for the Advancement of Artificial Intelligence (AAAI)}{[\url{http://www.aifb.uni-karlsruhe.de/WBS/seb/publications/pronto-aaai07.pdf}]}

\cventry{11.2005}{Martin Bleichner, Eugenie Giesbrecht, Helmar Gust, Eva-Maria Leicht, Petra
  Ludewig, Sabine M{\"o}ller, Wiebke M{\"u}ller, Martin Schmidt, Moritz
  Stefaner, Egon Stemle, and Katja Wilke}{ASADO: The analysis and structuring of aviation documents -- final
  report}{Technical report, Institute of Cognitive Science at the University of
  Osnabr{\"u}ck and Institute of Applied Linguistics at the University of
  Hildesheim}{}{[\url{http://iiegn.de/static/doc/bleichneretal2005.pdf}]}

\cventry{10.2005}{Egon Stemle}{A Flexible Integration of CL Techniques for Analyzing Linguistic Units in Large Document Repositories (unpublished)}{Cognitive Science BSc Thesis}{Institute of Cognitive Science, University of Osnabr\"{u}ck, Germany}{[\url{http://iiegn.de/static/doc/stemle2005.pdf}]}
\closesection{}

%\pagebreak

\section{Administrative Experience}
	\cventry{04.2006 --\\-- 03.2007}{Member of the Examination Board (Pr\"{u}fungsausschuss)}{}{Cognitive Science Study Programme, University of Osnabr\"{u}ck}{}{}
	\cventry{12.2005 --\\-- 02.2006}{Member of the Search Committee (Besetzungkommission)}{}{Institute of Cognitive Science}{University of Osnabr\"{u}ck}{BAT IIa research associate in Artificial Intelligence.}
	\cventry{04.2005 --\\-- 03.2008}{Member of the Academic Studies Commission (Studienkommission)}{}{Cognitive Science Study Programme, University of Osnabr\"{u}ck}{}{}
	\cventry{02.2005 --\\-- 01.2006}{Member of the Search Committee (Berufungskommission)}{}{Faculty of Humanities}{University of Osnabr\"{u}ck}{W3 professor ship Artificial Intelligence and Cognitive Science.}
	\cventry{10.2003 --\\-- 04.2008}{Member of the Steering Committee (Vorstand)}{}{Institute of Cognitive Science}{University of Osnabr\"{u}ck}{}
	\cventry{07.2003 --\\-- 10.2003}{Member of the Search Committee (Besetzungskommission)}{}{Institute of Cognitive Science}{University of Osnabr\"{u}ck}{BAT IIa research associate in Artificial Intelligence.}
\closesection{}

%\pagebreak

\section{Other Activities}
	\cventry{04.2006}{A for Alibi Symposium}{}{Uqbar Foundation}{Utrecht University Museum [\url{http://www.sternbergpress.com/?pageId=1204}]}{}
	\cventry{12.2005}{Amsterdam 2.0 Exhibition}{}{Mediamatic}{Amsterdam [\url{http://www.mediamatic.net/artefact-9850-en.html}]}{Technical counsel for Kasper Andreasen \& Tine Melzer}
	\cventry{03.2003}{Spring School}{}{Interdisciplinary College 2003 (IK2003 [\url{http://www.ik-guenne.de/html/ik2003.html}])}{G\"{u}nne at Lake M\"{o}hne}{Focus Theme: Applications, Brains and Computers}
	\cventry{2003}{The Complete Dictionary}{}{Tine Melzer}{26 volumes, A--Z}{Programming and Processing [\url{http://www.tinemelzer.eu}]}
	\cventry{since 11.2002}{Member of the German Cognitive Science Society (GK e.V.)}{}{[\url{http://www.gk-ev.de}]}{}{}
	\cventry{since 1995}{IT freelancer and technical consultant}{}{}{}{}
	\cventry{1992 --\\-- 1995}{Representative player for the Bavarian and the Southern German handball team}{}{}{}{}
	% \cventry{1993 \& 1994}{Bavarian champion in handball with the school team}{}{}{}{} %{Sigmund-Schuckert-Gymnasium}{N\"{u}rnberg}{}{}

    %\cventry{February 2006--\\current}{Maintainer of the a CTAN package}{CTAN}{World}{}{Maintainer of the {\ttfamily moderncv} package, meant to ease the production of beautiful curriculum vit\ae{}s.}
    %\cventry{2005--2006}{Mathematics tutor}{UCL}{Louvain-la-Neuve}{}{Supervision of practical sessions for a mathematical course given to second year engineering students (course \emph{FSAB1104: Numerical Methods}).\hfill{\itshape\color{see}\footnotesize{}See \httplink{www.legat-online.be/b2q1/num}.}}
    %\cventry{2004--2006}{Cultural project leader}{Tchouque-Tschouk Kot}{Louvain-la-Neuve}{}{Leader of a student home with a cultural project, requiring day to day management as well as the organization of public events.\hfill{\itshape\color{see}\footnotesize{}See \httplink{www.organe.be}.}}
    %\cventry{1999--2001}{IMO preselected}{SBPMef}{Wépion}{}{Advanced mathematical training, as Belgian preselected candidate for the International Mathematical Olympiads, selected by the Belgian mathematical society.\hfill{\itshape\color{see}\footnotesize{}See \weblink{imo.math.ca/belgium.html}.}}
\closesection{}

%\pagebreak{}

\section{Languages}
\cvlanguage{German}{Native}{}
\cvlanguage{English}{Excellent}{Primary education language at university.}
\cvlanguage{French}{Basic}{5 years training during high-school.}
%\cvlanguage{Dutch}{Basic}{Primary education degree obtained in a Dutch college\\(Sint-Jozefscollege te Sint-Pieters-Woluwe).}
\closesection{}

%\pagebreak{}

\section{Computer Skills}
\cvcomputer{programming}{\textsc{Java}, Prolog, Python, ML}{scripting}{Bash, PHP}
\cvcomputer{scientific}{Matlab, Octave, R, PDP++, Weka, Prot\'{e}g\'{e}, OmniFind+UIMA}{design}{\LaTeX, InDesign, Illustrator, OmniGraffle}
\cvcomputer{OS}{Linux, OS X, Windows 2000/XP}{service}{Apache, PHP, MySQL, PostgreSQL, Exim, Mailman, SpamAssassin, DSpam, Trac, Subversion, ISC DHCP, ISC Bind, Cyrus, Dovecot, OpenVPN, Shorewall, Darwin Calendar Server}

%\cvcomputer{CM}{Drupal, Mediawiki, Wordpress}
\closesection{}

%\section{Interests}
%\cvitem{design}{\small I am a design fan, especially when it comes to typography and photography.}
%\cvitem{adventure sports}{\small I like practicing adventure sports like skiing, rock climbing and scuba diving, and have been a boy scout for five years.}
%\cvitem{travelling}{\small I have been living abroad during my childhood, and love travelling around the world.}
%\closesection{}

%\section{Section with a list}
%\cvlistitem{Single item.}
%\cvlistitem{Another single item.}
%\cvlistdoubleitem{Double\dots{}}{\dots{} item.}
%\cvlistdoubleitem{Another double\dots{}}{\dots{} item.}
%\closesection{}

%\section{Section with your own content}\closesection
%Your content here, inside the normal \LaTeX{} environment. You can use any regular \LaTeX{} command, display mathematics
%\[e =m\,c^2,\]
%put some table or figure, \dots

%\emptysection{}
%\cvitem{Now}{Back to moderncv layout, without making a new section :-)}

%\nocite{*}
%\bibliographystyle{plain}
%\bibliography{jdoe_publications}
\end{document}
