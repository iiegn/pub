Egon Stemle is a researcher in the Institute for Applied Linguistics at Eurac Research, Bolzano, Italy.  As a cognitive scientist, he passionately combines the humanistic and analytical methods of the arts and formal sciences, focusing on the area where computational linguistics and artificial intelligence converge.

He works on the creation, standardisation, and interoperability of tools for editing, processing, and annotating linguistic data and enjoys working together with other scientists on their data but also collects or helps to collect new data from the Web, from computer-mediated communication and social media, and from language learners. He is an advocate of open science to make research and data available for others to consult or reuse in new research.

His curiosity in research is driven by the question why humans handle incomplete and -- more often than not -- inconsistent structured concepts just fine, whereas computational processes are often of little avail or fail completely. 

