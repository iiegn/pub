\documentclass[11pt,a4paper]{moderncv}

% moderncv themes
\moderncvtheme[blue]{classic}                 % optional argument are 'blue' (default), 'orange', 'red', 'green', 'grey' and 'roman' (for roman fonts, instead of sans serif fonts)
%\moderncvtheme[green]{classic}                % idem

% character encoding
\usepackage[utf8]{inputenc}                   % replace by the encoding you are using

\usepackage{url}
%% Define a new 'leo' style for the package that will use a smaller font.
\makeatletter
\def\url@leostyle{%
  \@ifundefined{selectfont}{\def\UrlFont{\sf}}{\def\UrlFont{\footnotesize\color{see}\itshape}}}

\makeatother
%% Now actually use the newly defined style.
\urlstyle{leo}

% personal data (the given example is exhaustive; just give what you want)
\firstname{egon~w.}
\familyname{stemle}
\title{Researcher}
\address{European Academy of Bozen/Bolzano (EURAC)\\Viale Druso, 1}{I-39100 Bolzano (BZ)}
\email{egon.stemle@eurac.edu}
%\phone{+39.0471.055.129}
\extrainfo{\url{http://www.eurac.edu/staff/estemle/}}
%\address{Center for Mind/Brain Sciences\\University of Trento\\Palazzo Fedrigotti\\Corso Bettini 31}{I-38068 Rovereto (TN)} % for classic style
%\email{egon.stemle@cimec.unitn.it}
%\address{Sedanstrasse~16, 49076~Osnabr\"{u}ck, Germany}  % for casual style
%\phone{+49.541.969-3362}
%\email{egon.stemle@uos.de}
%\extrainfo{\url{http://www.iiegn.de}}
\photo[64pt]{estemle} % also optional, and the optional argument is the height the picture must be resized to

%\renewcommand{\listsymbol}{{\fontencoding{U}\fontfamily{ding}\selectfont\tiny\symbol{'102}}} % define another symbol to be used in front of the list items

% the ConTeXt symbol
\def\ConTeXt{%
  C%
  \kern-.0333emo%
  \kern-.0333emn%
  \kern-.0667em\TeX%
  \kern-.0333emt}

% slanted small caps (only with roman family; the sans serif font doesn't exists :-()
%\usepackage{slantsc}
%\DeclareFontFamily{T1}{myfont}{}
%\DeclareFontShape{T1}{myfont}{m}{scsl}{ <-> cork-lmssqbo8}{}
%\usefont{T1}{myfont}{m}{scsl}Testing the font

% command and color used in this document, independently from moderncv 
\definecolor{see}{rgb}{0.5,0.5,0.5}% for web links
\newcommand{\up}[1]{\ensuremath{^\textsf{\scriptsize#1}}}% for text subscripts

%----------------------------------------------------------------------------------
%            content
%----------------------------------------------------------------------------------
\begin{document}
\maketitle


\section{}
Egon Stemle is a Cognitive Scientist: he studies skills like perception, thinking, learning, motor function, and language by combining the humanistic and analytical methods of the arts and the formal sciences.

His research focus within this relatively new `inter-discipline' lies in the area where Computational Linguistics and Artificial Intelligence converge. He is working on computer aided fabrication of ontologies from large document repositories, the technological feasibility thereof and the utilization of cross-linked structured data in applications, and on tools for editing, processing, and annotating linguistic data.

His driving force is the question why humans handle incomplete and -- more often than not -- inconsistent structured concepts just fine, whereas computational processes are often of little avail or fail completely. 



\section{Education}
    \cventry{02.2006 --\\-- 03.2009}{MSc in Cognitive Science}
        {with distinction}
        {}{University
        of Osnabr\"{u}ck, Germany}{Majors: \emph{Linguistics and Computational
        Linguistics} and \emph{Artificial Intelligence}, One Year Study Project:
        Analysis and Structure of Aviation Documents (ASADO),
        [\url{http://www.cogsci.uni-osnabrueck.de/~ASADO}]. }
    \cventry{03.2002 --\\-- 09.2002}{Semester abroad}{ILLC (Institute for
        Logic, Language and Computation)}{}{University of Amsterdam, The
        Netherlands}{}
    \cventry{10.2000 --\\-- 01.2006}{BSc in Cognitive Science}{}{}{University
        of Osnabr\"{u}ck, Germany}{}
    \cventry{12.1998 --\\-- 07.2000}{Apprenticeship as Travel
        Agent}{}{}{WORLDWIDE Touristik, N\"{u}rnberg, Germany}{}
    \cventry{06.1997}{Abitur}{}{}{Sigmund-Schuckert-Gymnasium, N\"{u}rnberg,
        Germany}{}
\closesection{}


\section{Master's Thesis}
    \cvitem{title}{\emph{Hybrid Sweeping: Streamlined Perceptual
        Structured-Text Refinement}}
    \cvitem{supervisors}{Prof.~Dr.~Stefan Evert and Prof.~Dr.~Peter K\"{o}nig}
    \cvitem{description}{\small Development of a perceptually driven content
        extraction architecture for Web pages -- exemplified by a Web page cleaning
        system.}
\closesection{}


%\section{Bachelor's Thesis}
%    \cvitem{title}{\emph{A Flexible Integration of CL Techniques for Analyzing
%        Linguistic Units in Large Document Repositories}}
%    \cvitem{supervisors}{PD Petra Ludewig and PD Helmar Gust}
%    \cvitem{description}{\small The basic computational linguistic processing
%        stages from raw text documents towards a linguistically enriched
%        annotation for documents are presented.}
%\closesection{}
%
%
%\pagebreak
\section{Research and Professional Experience}
	%\cventry{date}{main title}{three}{four}{five}{six}
	\cventry{since 02.2012}
        {Researcher}
        {Institute for Specialised Communication and Multilingualism at the
        European Academy of Bozen/Bolzano (EURAC) [\href{http://www.eurac.edu}
        {$\rightarrow$EURAC}]}
        {}{}{}
	\cventry{04.2009 --\\-- 01.2012}
        {Research Fellow}
        {LiveMemories sponsored grant, financed by the Provincia Autonoma of
        Trento, for research activities at the Center for Mind/Brain Sciences
        (CIMeC) of the University of Trento [\href{http://www.livememories.org}
        {$\rightarrow$LiveMemories},
        \href{http://www.cimec.unitn.it}{$\rightarrow$CIMeC}]}
        {}{}{}
    \cventry{since 2009}
        {Technology Consultant and Shareholder}
        {Whitematter Labs [\url{http://whitematterlabs.com}]}
        {\small provider of neuroscientific technologies for marketers}
        {}{}
    \cventry{04.2008 --\\-- 2010}
        {Project Member}
        {GoodGaze [\url{http://www.goodgaze.com}]}
        {\small research project to predict where people will look on Web pages}
        {}{}
    \cventry{2008}
        {Freelancer}
        {ontoprise GmbH [\url{http://www.ontoprise.de}]}
        {\small Integrating in-house technology with IBM's OmniFind;
            Feasibility Study: evaluate to which extent very specific
            information extraction from large web data collections combined with
            semi-automatic cleaning is technically feasible, what benefits and
            efforts are to be expected -- while also considering external
        alternatives}
        {}{}
    \cventry{10.2006 --\\-- 03.2008}
        {Research Participant}
        {DAAD funded research project (Project-linked exchange of academics and
        scientists, PPP)}
        {\small collaboration between the Institute of Cognitive Science,
        U.~Osnabr\"{u}ck, Germany, the CNRS UMR 8163 (Savoirs, Textes,
        Langages), U.~of Lille (III), France, and the Departament de
        Traducci\'{o} i Filologia, U.~Barcelona (Pompeu Fabra), Spain}
        {\small title: Reference to Abstract Objects in Natural Language
        (OntoRef) [\url{http://parles.upf.es/glif/pub/abstract/index.html}]}
        {}
    \cventry{04.2006 --\\-- 03.2008}
        {Research Assistant}
        {Institute of Applied Informatics and Formal Description Methods,
        University of Karlsruhe, Germany} {\small Knowledge Management Research
        Group}
        {[\url{http://www.aifb.uni-karlsruhe.de/Forschungsgruppen/WBS/english}]}
        {Maintenance and development of software and scientific counsel;
        e.g.~evaluating CL tools for OmniFind+UIMA and hooking them together.}
    \cventry{08.2004 --\\-- 09.2005}
        {Research Participant}
        {cooperation between the Universities of Osnabr\"{u}ck and Hildesheim,
        and the aircraft manufacturer AIRBUS}
        {\small title: ASADO -- project to research methodologies and
        technologies to analyze and structure the huge amount of documentation
        produced during aircraft construction}
        {}{}
    \cventry{08.2003 --\\-- 12.2003}
        {Student Assistant}
        {Institute of Cognitive Science, University of Osnabr\"{u}ck, Germany}
        {\small Artificial Intelligence Research Group}
        {[\url{http://www.cogsci.uni-osnabrueck.de/~ai/milca.html}]}
        {Maintenance and development of software for the MiLCA (Media intensive
        learning units for courses in computational linguistics) project.}
    \cventry{10.2001 --\\-- 03.2009}
        {System Administrator}
        {Institute of Cognitive Science, University of Osnabr\"{u}ck, Germany}
        {}
        {}
        {Maintenance of computer hardware and software, server administration
        (Web Services, Databases, File Sharing, Printing, Mail, Calendaring,
        (D)VCS, CMS), user support; assisting work groups in making strategic
        decisions about hardware and software.}
    %\cventry{11.1997 --\\-- 11.1998}
    %    {Alternative Civilian Service}
    %    {Alten-, Wohn- und Pflegeheim Reichelsdorf, and Fahrdienst f\"{u}r
    %    Behinderte gGmbH; N\"{u}rnberg, Germany}
    %    {}{}{}
\closesection{}


%\pagebreak
\section{Teaching Experience}
    \cventry{2009 -- 2011\\(yearly)}
        {Lecture on Web Corpora}
        {Course in text processing}
        {offered to students of the School of Humanities and Philosophy, of the
        International Master in Cognitive Science and of the Master in Human
        Language Technologies and Interfaces}
        {coordinated by Marco Baroni, University of Trento}
        {}
    \cventry{10.2003 --\\-- 02.2006\\(every other semester)}
        {Artificial Intelligence Tutor}
        {Methods of Artificial Intelligence}
        {Second year Cognitive Science class}
        {Institute of Cognitive Science, University of Osnabr\"{u}ck}
        {Weekly tutoring of the lecture material, assessment of programming
        assignments, homework corrections, and exam preparation.}	
    \cventry{04.2003 --\\-- 07.2006\\(every other semester)}
        {Computational Linguistics Tutor}
        {Introduction to Computational Linguistics}
        {First year Cognitive Science class}
        {Institute of Cognitive Science, University of Osnabr\"{u}ck}
        {Weekly tutoring of the lecture material, preparation and assessment of
        programming assignments, homework corrections, and exam preparation.}
    \cventry{10.2002 --\\-- 02.2003}
        {Theoretical Neuroscience Tutor}
        {Introduction to Theoretical Neuroscience}
        {Second year Cognitive Science class}
        {Institute of Cognitive Science, University of Osnabr\"{u}ck}
        {Weekly tutoring of the lecture material, homework corrections.}
    \cventry{10.2001 --\\-- 02.2002}
        {Computer Science Tutor}
        {Algorithms}
        {First year Computer Science class}
        {Computer Science Department, University of Osnabr\"{u}ck}
        {Weekly oral assessment of 10 2-people groups and homework corrections.}
\closesection{}



%\nocite{*}
%\bibliographystyle{plain}
%\bibliography{CV.bib}
% see CV_en.bbl and modify cventries accordingly...


%\pagebreak
\section{Publications}
    %\cventry{}
    %    {}
    %    {\small}
    %    {\small}
    %    {\small}
    %    {}
    %
    % PAISA' www.corpusitaliano.it
    %\cventry{11.2013 \\ (Interview / Beitrag)}
    %    {Egon Stemle, Alexander Onysko}
    %    {Academia 64 - Language as a Detective Story}

    %\cventry{11.2013 \\ (Submitted)}
    %    {Lionel Nicolas, Egon Stemle, Aivars Glaznieks, Andrea Abel}
    %    {A Generic Data Workflow for Building Annotated Text Corpora}
    %    {proceedings of the conference "Compiling and using learner corpora to
    %    teach and assess productive and interactive skills in foreign
    %    languages at university level"
    %    Peter Lang (Linguistic Insights series)
    
    %\cventry{10.2013 \\ (Submitted)}
    %    {Verena Lyding, Lionel Nicolas and Egon Stemle}
    %    {‘interHist’ ̶ an interactive visual interface for corpus exploration}
    %
    %    {Andrea Abel, Aivars Glaznieks, Lionel Nicolas, Egon Stemle}
    %    {KoKo: An L1 learner corpus for German}
    %
    %    {Egon W. Stemle}
    %    {Canola Corpus - A Gold Standard for Web Page Cleaning}
    %
    %    {LREC 2014}

    %\cventry{09.2013 \\ (Interview)}
    %    {Egon Stemle}
    %    {\small WaC \& STirWaC}
    %    {\small http://www.eurac.edu/en/newsevents/latest/NewsDetails.html?entryid=135442}
    %    {\small EURAC, Bolzano/Bozen, IT}
    %    {}

    \cventry{(in Preparation)}
        {Egon Stemle, Alexander Onysko}
        {\small The role of transfer in automated L1 identification of English learner essays}
        {\small in \emph{Peukert, Hagen (ed.): Transfer Effects in Multilingual Language Development}}
        {\small John Benjamins, Amsterdam \& Phildadelphia}
        {}
    
    \cventry{(in Preparation)} 
        {Aivars Glaznieks, Andrea Abel, Verena Lyding, Lionel Nicolas, Egon
        Stemle}
        {\small Establishing a Standardised Procedure for Building Learner
        Corpora – a Response to Demands and Suggestions of Users}
        {\small Proceedings of Learner Language, Learner Corpora 2012
        (LLLC2012), Oct 5-6, University of Oulu, Finland}
        {\small Jarmo Jantunen, Sisko Brunni \&
        Marianne Spoelman (eds.). Louvain-la-Neuve: Presses universitaires de
        Louvain}
        {}

    \cventry{(in Preparation)} 
        {Aivars Glaznieks, Egon Stemle}
        {\small Challenges of building a CMC corpus for analyzing writer's
        style by age: The DiDi project} 
        {\small Special Issue of Journal of Language Technology and
        Computational Linguistics: Building and annotating corpora of
        computer-mediated discourse. Issues and Challenges at the Interface of
        Corpus and Computational Linguistics}
        {\small Michael Beißwenger, Nelleke Oostdijk, Angelika Storrer \& Henk van den Heuvel (eds.)}
        {}

    % RANLP-2013 Young Researcher Award
    % ?also proceedings
    \cventry{09.2013}
        {Lionel Nicolas, Egon~W.~Stemle, Klara Kranebitter, Verena Lyding}
        {\small High-accuracy phrase translation acquisition through
        battle-royale selection}
        {\small Proceedings of Recent Advances in Natural Language Processing,
        RANLP 2013}
        {\small Sep 7-13, Hissar, Bulgaria}
        {}

    %07.2013
    %    Tine Melzer, Egon Stemle 
    %    Ludwig \& Gertrude
    %    Supplement to the PhD research Ludwig \& Gertrude Meeting in Language, Faculty of Arts, Planetary Collegium, M-node, 2013

    \cventry{07.2013}
        {Stefan Evert, Egon Stemle, Paul Rayson (eds.)}
        {\small Proceedings of the 8th Web as Corpus Workshop (WAC-8)}
        {\small Workshop at the seventh international Corpus Linguistics conference (CL2013)}
        {\small Jul 22, Lancaster, UK}
        {}

    %\cventry{06.2013 \\ (Interview)}
    %    {Egon Stemle}
    %    {\small Mi piace ;-)}
    %    {\small Science 4 You, Academia no.62}
    %    {\small EURAC, Bolzano/Bozen, IT}
    %    {}
    %
    %\cventry{06.2013} \\ (Interview)}
    %    {Egon Stemle}
    %    {\small A nuie Dimension in dor Sprochforschung}
    %    {\small EURAC Activity Report 2012/13}
    %    {\small EURAC, Bolzano/Bozen, IT}
    %    {}

    \cventry{06.2013}
        {Klara Kranebitter, Egon~W.~Stemle}
        {\small Constructing concept relation maps to support building concept
        systems in comparative legal terminology}
        {\small Terminology \& Ontology: Theories and applications (TOTh'2013)}
        {\small Jun 6-7, Chamb\'{e}ry, France}
        {}

    \cventry{09.2012}
        {Lionel Nicolas, Egon~W.~Stemle, Klara Kranebitter}
        {\small Towards high-accuracy bilingual phrase acquisition from parallel corpora}
        {\small in {\em Proceedings of KONVENS 2012} (LexSem 2012 workshop), Vienna, Austria}
        {\small \"{O}sterreichische Gesellschaft f\"{u}r Artificial Intelligence (\"{O}GAI)}
        {}
    
    \cventry{07.2012}
        {Francesca Bonin, Fabio Cavulli, Massimo Poesio, Egon~W.~Stemle}
        {\small Annotating Archaeological Texts: An Example of
        Domain-Specific Annotation in the Humanities}
        {\small in {\em Proceedings of the Sixth Linguistic Annotation
        Workshop, ACL2012}, Jeju, Republic of Korea}
        {\small Association for Computational Linguistics}
        {}

    %\cventry{02.2012 \\ Participant}
    %    {CLARIN-AT - DARIAH-AT -- Europäische Forschungsinfrastrukturen in den
    %    Geisteswissenschaften}
    %    {Institut für Corpuslinguistik und Texttechnologie (ICLTT) der
    %    Österreichischen Akademie der Wissenschaften, Zentrum für
    %    Informationsmodellierung (ZIM) der Universität Graz}
    %    {21./22.02.2012}
    %    {}
    %    {}

    \cventry{11.2011}
        {Asif Ekbal, Francesca Bonin, Sriparna Saha, Egon Stemle, Eduard Barbu,
        Fabio Cavulli, Christian Girardi, Massimo Poesio}
        {\small Rapid Adaptation of NE Resolvers for Humanities Domains using
        Active Annotation}
        {\small in {\em Journal for Language Technology and Computational Linguistics (JLCL)},
        26(2):39--51, 2011}
        {}
        {}

    \cventry{11.2011}
        {Massimo Poesio, Eduard Barbu, Francesca Bonin, Fabio Cavulli, Asif
        Ekbal, Egon Stemle, Christian Girardi}
        {\small The Humanities Research Portal: Human Language Technology Meets
        Humanities Publication Archives}
        {\small in {\em SDH2011 - Supporting Digital Humanities: Answering the
        unaskable}, Copenhagen, DK}
        {}
        {}

    \cventry{07.2011}
        {Brian Murphy, Egon~W.~Stemle}
        {\small PaddyWaC: A Minimally-Supervised Web-Corpus of Hiberno-English}
        {\small in {\em Proceedings of the First Workshop on Algorithms and
        Resources for Modelling of Dialects and Language Varieties
        (DIALECTS2011)}, Edinburgh, Scotland, UK}
        {\small Association for Computational Linguistics}
        {}

    \cventry{06.2011}
        {Massimo Poesio, Eduard Barbu, Egon~W.~Stemle, Christian Girardi}
        {Structure-Preserving Pipelines for Digital Libraries}
        {\small in {\em Proceedings of the 5th ACL-HLT Workshop on Language
        Technology for Cultural Heritage, Social Sciences, and Humanities
        (LaTeCH 2011)}, Portland, OR, USA}
        {\small Association for Computational Linguistics}
        {}

    \cventry{05.2010}
        {Kepa~J. Rodr{\'i}guez, Francesca Delogu, Jannick Versley,
        Egon~W.~Stemle, Massimo Poesio} {Anaphoric Annotation of Wikipedia
        and Blogs in the Live Memories Corpus}
        {\small in {\em Proceedings of the 7th Conference on International
        Language Resources and Evaluation (LREC'10)}, Valletta, Malta}
        {\small European Language Resources Association (ELRA)}
        {}

    \cventry{09.2009}
        {Johannes~M.~Steger, Egon~W.~Stemle}
        {{KrdWrd} -- The Architecture for unified Processing of Web Content}
        {\small in {\em Proceedings of the 5th Web as Corpus Workshop (WAC5)}}
        {\small Pre-SEPLN Workshop, Donostoia-San Sebasti{\'a}n, Basque
        Country, Spain}
        {[\url{http://www.sigwac.org.uk/raw-attachment/wiki/WAC5/WAC5_proceedings.pdf}]}

    \cventry{09.2007}
        {Daniel Bauer, Judith Degen, Xiaoye Deng, Priska Herger, Jan Gasthaus,
        Eugenie Giesbrecht, Lina Jansen, Christin Kalina, Thorben Kr{\"u}ger,
        Robert M{\"a}rtin, Martin Schmidt, Simon Scholler, Johannes Steger,
        Egon Stemle, Stefan Evert}
        {FIASCO: Filtering the Internet by automatic Subtree Classification,
        Osnabr{\"u}ck}
        {\small in {\em Building and Exploring Web Corpora. Proceedings of the
        3rd Web as Corpus Workshop, incorporating CLEANEVAL (WAC3-2007)}}
        {\small Presses universitaires de Louvain}
        {[\url{http://purl.org/stefan.evert/PUB/BauerEtc2007_FIASCO.pdf}]}
    
   \cventry{07.2007}
        {Sebastian Blohm, Philipp Cimiano, Egon Stemle}
        {Harvesting Relations from the Web - Quantifiying the Impact of
        Filtering Functions}
        {\small in {\em Proceedings of the 22nd Conference on Artificial
        Intelligence (AAAI-07)}}
        {\small Association for the Advancement of Artificial Intelligence
        (AAAI)}
        {[\url{http://www.aifb.uni-karlsruhe.de/WBS/seb/publications/pronto-aaai07.pdf}]}

    \cventry{11.2005}
        {Martin Bleichner, Eugenie Giesbrecht, Helmar Gust, Eva-Maria Leicht,
        Petra Ludewig, Sabine M{\"o}ller, Wiebke M{\"u}ller, Martin Schmidt,
        Moritz Stefaner, Egon Stemle, Katja Wilke}
        {ASADO: The Analysis and Structuring of Aviation Documents -- Final Report}
        {\small Technical Report, Institute of Cognitive Science at the
        University of Osnabr{\"u}ck and Institute of Applied Linguistics at the
        University of Hildesheim}
        {}
        {[\url{http://iiegn.de/static/doc/bleichneretal2005.pdf}]}

    %\cventry{10.2005}
    %    {Egon Stemle}
    %    {A Flexible Integration of CL Techniques for Analyzing Linguistic Units in Large Document Repositories (unpublished)}
    %    {\small Cognitive Science BSc Thesis}
    %    {\small Institute of Cognitive Science, University of Osnabr\"{u}ck, Germany}
    %    {[\url{http://iiegn.de/static/doc/stemle2005.pdf}]}
\closesection{}


%\pagebreak
\section{Talks}
    \cventry{22.11.2013}
        {Andrea Abel, Aivars Glazniek, Egon~W.~Stemle}
        {\small Automatische Annotation von Schülertexten -- Herausforderungen
        und Lösungsvorschläge am Beispiel des Projekts KoKo}
        {\small Arbeitsgruppe: Korpusbasierte Linguistik} 
        {\small Workshop at the 40.~Österreichische Linguistiktagung, Nov
        22-24, Universität Salzburg, Salzburg, Austria}
        {}

    \cventry{11.2013}
        {Verena Lyding*, Claudia Borghetti*, Henrik Dittmann, Lionel Nicolas, Egon Stemle}
        {\small Open Corpus Interface for Italian Language Learning}
        {\small 6th edition of the International Conference ICT for Language Learning}
        {\small Nov 14-15, Florence, Italy}
        {}

    \cventry{23.09.2013}
        {Aivars Glazniek, Egon~W.~Stemle}
        {\small Herausforderungen bei der automatischen Verarbeitung von dialektalen IBK-Daten}
        {\small GSCL-Workshop \emph{Verarbeitung und Annotation von Sprachdaten aus
        Genres internetbasierter Kommunikation}}
        {\small Workshop at the International Conference of the German Society
        for Computational Linguistics and Language Technology, Sep 23,
        Darmstadt, Germany}
        {}

    \cventry{24.06.2013}
        {Egon~W.~Stemle, Verena Lyding}
        {\small The future of BootCaT: A Creative Commons License filter}
        {\small BootCaTters of the world unite! (BOTWU), A workshop (and a
        survey) on the BootCaT toolkit}
        {\small Jun 24, Department of Interpreting and Translation, University
        of Bologna Forl\`{i}, Italy}
        {}

    \cventry{05.2013}
        {Lionel Nicolas*, Aivars Glaznieks, Egon~W.~Stemle, Andrea Abel}
        {\small Compiling and using a German learner corpus: the Koko project}
        {\small Compiling and Using Learner Corpora to teach and assess
        productive and interactive skills in foreign languages at university
        level - Learner Corpora 2013}
        {\small May 16-17, Padova, Italy}
        {}

    \cventry{03.2013}
        {Aivars Glaznieks*, Egon Stemle, Andrea Abel, Verena Lyding}
        {\small Herausforderungen bei der Erstellung eines L1-Lernerkorpus:
        Lösungsvorschläge aus dem Projekt "KoKo"}
        {\small DGfS 2013 Workshop on Modellierung nicht-standardisierter
        Schriftlichkeit}
        {\small Workshop at the 35th Annual Conference of the German Linguistic
        Society (DGfS 2013), Mar 12-15, Potsdam, Germany}
        {}

    \cventry{03.2013}
        {Verena Lyding*, Lionel Nicolas, Egon Stemle}
        {\small interHist -- an interactive visualization for statistically
        enhanced query structures}
        {\small DGfS 2013 Workshop on the Visualization of Linguistic Patterns}
        {\small Workshop at the 35th Annual Conference of the German Linguistic
        Society (DGfS 2013), Mar 12-15, Potsdam, Germany}
        {}

    \cventry{13.02.2013}
        {Egon~W.~Stemle, Aivars Glaznieks}
        {\small Technical Aspects in Harvesting Data from Social Network Sites}
        {\small International workshop: Building Corpora of Computer-Mediated
        Communication: Issues, Challenges, and Perspectives}
        {\small Feb 13-15, Department of German Language and Literature, Faculty of
        Culture Studies, TU Dortmund University (Germany)}
        {}

    \cventry{02.2013}
        {Aivars Glaznieks*, Egon~W.~Stemle}
        {\small The Project DIDI. Writing on Social Network Sites -- A
        Corpus-based Observation of the Current Language Use in South Tyrol,
        with Particular Consideration of the Writers’ Age} 
        {\small International workshop: Building Corpora of Computer-Mediated
        Communication: Issues, Challenges, and Perspectives}
        {\small Feb 13-15, Department of German Language and Literature, Faculty of
        Culture Studies, TU Dortmund University (Germany)}
        {}

    \cventry{13.07.2012 \\ (Plenary Talk)}
        {Egon~W.~Stemle}
        {Web Corpus Creation and Cleaning}
        {\small Computer Applications in Linguistics: Student Research Workshop (CSRW2012)}
        {\small English Corpus Linguistics Group at the Institute of
        Linguistics and Literary Studies, Technische Universit\"{a}t Darmstadt}
        {}

    \cventry{04.05.2012 \\ (Implusreferat)}
        {Egon~W.~Stemle, Verena Lyding, Lionel Nicolas}
        {\small On visual Approaches towards Corpus Exploration}
        {\small 3rd Workshop of the Academic Network on "Internet Lexicography"}
        {\small}
        {}

\closesection{}


%\pagebreak
\section{Workshops and Conferences}
    \cventry{07.2013}
        {Organising Committee and Programme Committee}
        {Workshop}
        {8th Web as Corpus Workshop (WAC-8)}
        {[\url{http://sigwac.org.uk/wiki/WAC8}]}
        {Workshop day at the seventh international Corpus Linguistics
        conference (CL2013), Lancaster University, UK}

	\cventry{09.2006}
        {Student Worker at OTT06}
        {Workshop}
        {Ontologies in Text Technology: Approaches to Extract Semantic
        Knowledge from Syntactic Information.}
        {[\url{http://www.cogsci.uni-osnabrueck.de/~ott06}]}
        {jointly organized by the Institute of Cognitive Science at the
        University of Osnabr\"{u}ck and the project C2 of the distributed
        DFG-research group Text Technological Modelling of Information}

	\cventry{06.2006}
        {Student Worker at QITL-2}
        {Workshop}
        {Second Workshop on Quantitative Investigations in Theoretical
        Linguistics.}
        {[\url{http://www.cogsci.uni-osnabrueck.de/~qitl}]}
        {organized by the Computational Linguistics Group at the Institute of
        Cognitive Science}

	\cventry{09.2003}
        {Student Worker at EuroCogSci 2003}
        {Conference}
        {[\url{http://www.eurocogsci03.uni-osnabrueck.de}]}{First European
        Cognitive Science Meeting.}
        {jointly organized by the Cognitive Science Society and the German
        Cognitive Science Society}
\closesection{}


%\pagebreak
\section{Administrative Experience}
    \cventry{since 08.2012}
        {Secretary of the Special Interest Group on the Web as Corpus (SIGWAC)}
        {}
        {SIG of the Association for Computational Linguistics}
        {}
        {}
	\cventry{04.2006 --\\-- 03.2007}
        {Member of the Examination Board (Pr\"{u}fungsausschuss)}
        {}
        {Cognitive Science Study Programme, University of Osnabr\"{u}ck}
        {}
        {}
    \cventry{12.2005 --\\-- 02.2006}
        {Member of the Search Committee (Besetzungkommission)}
        {}
        {Institute of Cognitive Science}
        {University of Osnabr\"{u}ck}
        {BAT IIa research associate in Artificial Intelligence.}
    \cventry{04.2005 --\\-- 03.2008}
        {Member of the Academic Studies Commission (Studienkommission)}
        {}
        {Cognitive Science Study Programme, University of Osnabr\"{u}ck}
        {}
        {}
    \cventry{02.2005 --\\-- 01.2006}
        {Member of the Search Committee (Berufungskommission)}
        {}
        {Faculty of Humanities}
        {University of Osnabr\"{u}ck}
        {W3 professor ship Artificial Intelligence and Cognitive Science.}
    \cventry{10.2003 --\\-- 03.2008}
        {Member of the Steering Committee (Vorstand)}
        {}
        {Institute of Cognitive Science}
        {University of Osnabr\"{u}ck}
        {}
    \cventry{07.2003 --\\-- 10.2003}
        {Member of the Search Committee (Besetzungskommission)}
        {}
        {Institute of Cognitive Science}
        {University of Osnabr\"{u}ck}
        {BAT IIa research associate in Artificial Intelligence.}
\closesection{}


%\pagebreak
\section{Other Activities}
	\cventry{04.2006}
        {A for Alibi Symposium}
        {}
        {Uqbar Foundation}
        {Utrecht University Museum [%
        \href{http://www.sternbergpress.com/?pageId=1204}{link}]}
        {}	
    \cventry{12.2005}
        {Amsterdam 2.0 Exhibition}
        {}
        {Mediamatic}
        {Amsterdam [\href{http://www.mediamatic.net/artefact-9850-en.html}%
        {link}]}
        {Technical counsel for Kasper Andreasen \& Tine Melzer}
    \cventry{03.2003}
        {Spring School}
        {}
        {Interdisciplinary College 2003 (IK2003 [%
        \href{http://www.ik-guenne.de/html/ik2003.html}{link}])}
        {G\"{u}nne at Lake M\"{o}hne}
        {Focus Theme: Applications, Brains and Computers}
    \cventry{2003}
        {The Complete Dictionary}
        {}
        {Tine Melzer}
        {26 volumes, A--Z}
        {Programming and Processing
        [\href{http://www.tinemelzer.eu/works/the-complete-dictionary/}
        {link}]}
    \cventry{since 11.2002}
        {Member of the German Cognitive Science Society (GK e.V.)}
        {}
        {}
        {[\href{http://www.gk-ev.de}{link}]}
        {}
    % \cventry{since 1995}{IT Freelancer and Technical Consultant}{}{}{}{}
	% \cventry{1992 --\\-- 1995}{Representative player for the Bavarian and the Southern German handball team}{}{}{}{}
	% \cventry{1993 \& 1994}{Bavarian champion in handball with the school team}{}{}{}{} %{Sigmund-Schuckert-Gymnasium}{N\"{u}rnberg}{}{}
\closesection{}


%\pagebreak{}
\section{Languages}
    \cvlanguage{German}{Native}{}
    \cvlanguage{English}{Excellent}{Main education language at university.}
    \cvlanguage{French}{Basic}{5 years training during high-school.}
    \cvlanguage{Italian}{Basic}{}
    %\cvlanguage{Dutch}{Basic}{Primary education degree obtained in a Dutch college\\(Sint-Jozefscollege te Sint-Pieters-Woluwe).}
\closesection{}


%\pagebreak{}
\section{Computer Skills}
    \cvcomputer{programming}
        {Python, Java, JavaScript (ECMA\-Scrip), Prolog, ML}
    {scripting}
        {Bash, Perl, PHP}
    \cvcomputer{scientific}
        {R, CQP, Oracle/Sun Grid Engine, Apache Hadoop, Matlab, OmniFind+UIMA,
        PDP++, Octave, Prot\'{e}g\'{e}}
    {design}
        {\LaTeX, gnuplot, Inkscape, Scribus, InDesign, Illustrator, Photoshop, GIMP}%, OmniGraffle}
    %\cvcomputer{OS}{Linux, OS X, Windows 2000/XP}{service}{LAMP web server, mail server}
    %\cvcomputer{OS}{Linux, OS~X,\\ Windows~2000/XP/Vista}{services}{Apache, MySQL, PostgreSQL, Exim, Mailman, SpamAssassin, DSpam, Trac, Subversion, ISC DHCP, ISC Bind, Cyrus, Dovecot, OpenVPN, Shorewall, Darwin Calendar Server, OpenLDAP, NIS/YP, Kerberos+NFSv4}
    %\cvcomputer{CM}{Drupal, Mediawiki, Wordpress}
    %\cvcomputer{office}{Word, Excel, Access, Powerpoint, LibreOffice}
\closesection{} %
%\section{Interests}
%\cvitem{design}{\small I am a design fan, especially when it comes to typography and photography.}
%\cvitem{adventure sports}{\small I like practicing adventure sports like skiing, rock climbing and scuba diving, and have been a boy scout for five years.}
%\cvitem{travelling}{\small I have been living abroad during my childhood, and love travelling around the world.}
%\closesection{}
%
%\section{Section with a list}
%\cvlistitem{Single item.}
%\cvlistitem{Another single item.}
%\cvlistdoubleitem{Double\dots{}}{\dots{} item.}
%\cvlistdoubleitem{Another double\dots{}}{\dots{} item.}
%\closesection{}
%
%\section{Section with your own content}\closesection
%Your content here, inside the normal \LaTeX{} environment. You can use any regular \LaTeX{} command, display mathematics
%\[e =m\,c^2,\]
%put some table or figure, \dots
%
%\emptysection{}
%\cvitem{Now}{Back to moderncv layout, without making a new section :-)}
%
%\nocite{*}
%\bibliographystyle{plain}
%\bibliography{jdoe_publications}
\end{document}
