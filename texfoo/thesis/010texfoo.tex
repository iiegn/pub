\documentclass[11pt,a4paper,pdftex,oneside,liststotoc,listsleft]{scrreprt}

% have koma type headings in rm font
\addtokomafont{sectioning}{\rmfamily}


%%%% package imports (order matters)

% specifies the encoding of this file (and \include or \input files)
\usepackage[latin1]{inputenc}

% in pdfTeX an active font can only refer to 256 glyphs at a time;
% select the std. T1 mapping for this document
\usepackage[T1]{fontenc}

% activate hyphenation
\usepackage[english]{babel}

% activate character protruding for margin kerning, 
% i.e. try to get a 'smoother' margin
\usepackage[activate]{pdfcprot}

% activate some symbols, e.g. \textmusicalnote (and more 'important' ones...)
\usepackage{textcomp}

%
% activate springer's minion and myriad font
\pdfmapfile{+springer.map}
\renewcommand{\sfdefault}{fmy}
\renewcommand{\rmdefault}{fmnx}
\renewcommand{\ttdefault}{lmtt}
%
% OR
%
% activate the Almost European computer modern font (cf. http://www.ctan.org/tex-archive/fonts/ae/)
%\usepackage{ae}

%
%\usepackage[right=7cm,left=2.5cm,top=2cm,bottom=3.5cm]{geometry}
\usepackage[top=3.0cm,bottom=4.0cm]{geometry}
\usepackage{setspace}

\usepackage{epic,eepic}
\usepackage{graphicx}
% this will produce a warning:
% LaTeX Warning: Command \@makecol has changed.
% seems to occur in combination with the setspace package.
\usepackage[stable, bottom]{footmisc}
%\usepackage{fullpage}
\usepackage{url}
\usepackage{amsmath}
\usepackage{amssymb}
\usepackage{tabularx}
\usepackage[pdftex]{color}
\usepackage[
	pdftex,
	final=true,
	pdfstartview=FitH
	]{hyperref}

%%%% hyper & options
\definecolor{myblue}{rgb}{0.25,0.25,0.75}
\definecolor{darkblue}{rgb}{0,0,0.75}
\definecolor{darkred}{rgb}{0.4,0,0}
\hypersetup{ 
colorlinks=true,
bookmarks=true,
bookmarksnumbered=true,
bookmarksopen=true,
bookmarksopenlevel=2,
pdftitle={\mypdftitle},
pdfauthor={\myauthor},
pdfsubject={\mytitle},
pdfkeywords={\mykeywords},
pdfproducer={pdflatex, inkscape, gnuplot},
frenchlinks=true,
pdfborder=0 0 0,
linkcolor=myblue,
%pagecolor=darkblue,
urlcolor=myblue,
citecolor=darkred,
setpagesize=true
}

%
% align numbering in TOC on the left side, i.e.
% 1
% 1.1
% 1.1.1
% ...
\usepackage{tocloft}
\usepackage{chngcntr}
\setlength{\cftchapnumwidth}{\cftsubsubsecnumwidth}
\setlength{\cftsecnumwidth}{\cftsubsubsecnumwidth}
\setlength{\cftsubsecnumwidth}{\cftsubsubsecnumwidth}
\setlength{\cftsubsubsecnumwidth}{\cftsubsubsecnumwidth}
\setlength{\cftsecindent}{0pt}
\setlength{\cftsubsecindent}{0pt}
\setlength{\cftsubsubsecindent}{0pt}

%%%% fancy & options
\usepackage{fancyhdr}
%\pagestyle{fancy}
\renewcommand{\footrulewidth}{0.5pt}
\renewcommand{\headrulewidth}{0.5pt}
\setlength{\headheight}{25pt}
\setlength{\headsep}{20pt}
\renewcommand{\chaptermark}[1]{\markboth{\quad #1}{\quad #1}}
\renewcommand{\sectionmark}[1]{\markright{#1}}
%\fancyhf{}
%\fancyfoot[CE]{\myauthor}
%\fancyfoot[CO]{\myStitle}
%\fancyhead[LE,RO]{\bfseries\thepage}
%\fancyhead[RE]{\bfseries\leftmark }
%\fancyhead[LO]{\bfseries\rightmark }

% do not reset footnote count on every chapter
%\counterwithout*{footnote}{chapter}

%%%% indexing options
%\setcounter{tocdepth}{3}
%\setcounter{secnumdepth}{3}
%\newcounter{lofdepth}
%\setcounter{lofdepth}{3}

%%%% new commands
\DeclareMathOperator{\project}{project}
\DeclareMathOperator{\CC}{CC}
\DeclareMathOperator{\NCC}{NCC}
\newcommand{\src}[1]{\texttt{#1}}
\newcommand{\fref}[1]{\src{#1} (c.f. \ref{#1})}
\newcommand{\email}[1]{\href{mailto:#1}{#1}}
\newcommand{\grad}[0]{^\circ}
\newcommand{\fig}[4]
{
 \begin{figure}[h]
  \centering
  \includegraphics[width=#1\textwidth]{../images/#2}
  \caption{#3}
  \label{#4}
 \end{figure}
}

%%%% typesetting options
\unitlength10mm
\renewcommand*{\tabularxcolumn}[1]{>{\small}m{#1}}


%\usepackage{natbib}
%\bibliographystyle{plainnat}
